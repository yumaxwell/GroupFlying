%# -*- coding: utf-8 -*-
%!TEX encoding = UTF-8 Unicode

%% Tex template of yumaxwell, version 0.4, Nov. 15th, 2016.
%% This is Mac chinese version template for article.

%% Test environment: MacTex2011(TexLive2011, Ghostscript9.02, Image
    %   Magick-Converter6.6.9)
%% Using compiler:  XeLaTex
%% For Chinese article.

%% Use Mac_CHN_Header.tex
\input{Header}
%%******************************************************************%%
%%                    Document header                               %%
%%******************************************************************%%
\begin{document}
%% Header and footer
\pagestyle{fancy}
\lhead{}
\chead{}
\rhead{}
\lfoot{}
\rfoot{}

\title{\yh 群体运动问题}
\author{Yumaxwell\thanks{Template version 0.3, DONOT COPY WITHOUT PERMISSION! CONTACT \underline{yuwenjun@pku.edu.cn}}}
%\date{2012-11-21}   % comment means today
\maketitle
\thispagestyle{empty} % this page has no header
%\tableofcontents        % content
%\listoffigures         % figures content
%\listoftables          % tables content

%%******************************************************************%%
%%******************************************************************%%
%% Main body

% Abstract of article
%\begin{abstract}

%\end{abstract}

\section{背景}
群体运动是自然界中常见的现象。对于一个群体,如鱼群、鸟群。每个个体可能仅具有比较简单的能力,但是集合在一起组成群体,能够表现出复杂的行为。

\section{问题描述}
给定一个群体,在三维空间中可以自由飞行,空间坐标采用笛卡尔坐标系$O-xyz$。群体中有$N$个独立个体,每个个体记为$N_i, i=1,2,\cdots,N$,其对应空间坐标为对于时间$t$的函数$\vec{x_i}(t) = [x_i(t),y_i(t),z_i(t)]$。
针对给定的初始群体位置$\vec{S}_1=\{\vec{x_1},\vec{x_2},\cdots,\vec{x_N}\}$,和预定的目标位置$\vec{S}_2=\{\vec{x_1}',\vec{x_2}',\cdots,\vec{x_N}'\}$,求解每个个体的飞行路径轨迹,以及花费的总体时间。

在飞行过程中,每个个体的速度和加速度\footnote[1]{或者开始阶段可以仅采用速度约束,但加速度约束最终也应增加。}在整个飞行过程中应该保持在限定范围内。
\begin{eqnarray}
	\vec{v_i}(t)&<& \vec{v_{\text{max}}} , \quad t\in[t_{\text{start}},t_{\text{end}}], i=1,2,\cdots,N. \label{e:alimDef} \\
	\vec{a_i}(t)&<& \vec{a_{\text{max}}} , \quad t\in[t_{\text{start}},t_{\text{end}}], i=1,2,\cdots,N. \label{e:vlimDef}
\end{eqnarray}
同时,在飞行过程中,任意两个个体的需要保持安全飞行间距,
\begin{equation}\label{e:dlimDef}
	\forall t \in [t_{\text{start}},t_{\text{end}}], \quad \forall i,j\in \{1,2,\cdots,N\}, i\ne j, \quad |\vec{x_i}(t)-\vec{x_j}(t)| \ge d_{\text{th}}.
\end{equation}

要求实现计算程序的模块化。

群体位置生成算法。

个体运动模型。

个体运动轨迹算法。

评估飞行安全距离。

图形显示。


\section{后续问题}
\begin{enumerate}
	\item 有哪些飞行策略?
	\item 如何找到时间最短的飞行策略?
	\item 是否有应用?比如缓解交通拥堵?
\end{enumerate}


%% Appendix
%\appendix{Appendix}

%% Reference

%\section*{References}
%\begin{thebibliography}{}
%	\bibitem{bib_source} Source reference.
%    \bibitem{bib_ref} reference name.
%\end{thebibliography}

%\bibliographystyle{ieee} % still have bugs here
%\bibliography{bibref}


\end{document}
%% End of article 